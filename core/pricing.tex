Stadia has 2 models:
\begin{enumerate}
    \item Stadia Pro.
    \item Stadia Base.
\end{enumerate}

\subsection{Stadia Pro}
Monthly subscription for \$9.99, let's you stream in 4K resolution and 5.1 surround sound.
Google selects you some games every now and then. \cite{stadiaPro}

Google priced `Stadia Pro' at \$9.99, which is a much lower price than its competitor Sony.
Sony - which owns `Playstation Now' an old competitor for Stadia but only on Playstation consoles - had its service priced at \$19.99. \cite{sonyPrices}

But after Google anounced `Stadia Pro' pricing, sony had to lower their prices to \$9.99.
Probably Google chose the `Penetration Pricing Strategy', and tried to draw attention from Sony's platform. \cite{pricingStategy}

\subsection{Stadia Base}
Free tier for the streaming, but you pay for the game which range between \$30 and \$60.
Not like Stadia Pro, with Stadia Base you will own the game.
Google caps the streaming quality of `Stadia Base' users to 1080p. \cite{stadiaBase}

Google is using `Stadia Base' as its `Freemium Pricing Strategy.'
It hopes that users will eventually pay to upgrade to `Stadia Pro' and use its full functionality.

With `Stadia Base', Google is also trying to fixing the issue of not owning the game - which a lot of people complained about.
Now you can own a game, Google will get revenue by acting as a store for those games, and can get customers for `Stadia Pro' using the freemium version.