\begin{figure}[h]
    \centering
    \includegraphics[width=0.8\textwidth]{images/reddit.png}
    \caption{Stadia Sub-Reddit Banner}
    \label{fig:reddit}
\end{figure}

To further understand users' demands and possible improvement, we have to explore a massive community for \emph{Google Stadia} users and enthusiasts. This community is \emph{Stadia sub-reddit} \cite{stadia_reddit}, which is the official community for Stadia users. \\

Aside from our analysis, we checked some recommendations and reported bugs from Stadia users. This information helps us to develop a list recommendations and improvements for Stadia service, in order to improve the overall marketing process. \\

Our recommendations focus on both promotion campaign and technical improvements, that can positively affect the market share of Google Stadia. We target the competition with game consoles, like \emph{PlayStation} and \emph{XBox}, and other cloud streaming services, like \emph{Nvidia Geforce Now} and \emph{Microsoft Project XCloud}. \\

Our recommendation list can be summarized as follows:
\begin{itemize}
    \item Stadia have to consider in-game promotions, as most gamers' opinions heavily rely on in-game recommendations. For example, \emph{Nvidia Geforce series} has been dominating the PC gaming market for years, due to in-game recommendations and the technologies, by which Nvidia enhances video games.
    \item Expanding the game library is a crucial thing for any gaming platform to flourish. Stadia has to expand its game library to include a wide range of video games from \texttt{AAA} games to \texttt{indies}.
    \item Stadia is only available through a limited number of platforms that support \emph{Google Chrome} or \emph{Chrome OS}. This can be a drawback for some users that don't have a compatible device. So, Stadia has to expand its compatibility to include a wide range of devices and applications, in order to increase the number of potential users.
    \item Technical issues can form a negative impression among users and enthusiasts and Stadia suffers from some issues during streaming and game rendering, especially for users with limited internet bandwidth. So, Google has to optimize its service for this type of users.
    \item As Google isn't a video game producer or publisher, it has cooperate with specialized video game publishers, like \emph{Blizzard Entertainment} and \emph{Bethesda Softworks}, in order to promote their platform and even have some exclusives on their own. This can positively affect their market share and attract more users to their platform.
\end{itemize}

